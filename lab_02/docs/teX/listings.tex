\chapter*{Код программы}

\begin{lstlisting}[label=code1, caption = Лабораторная работа №2]
from math import *
import matplotlib.pyplot as plt

table_I = [0.5, 1, 5, 10, 50, 200, 400, 800, 1200]
table_T0 = [6730, 6790, 7150, 7270, 8010, 9185, 10010, 11140, 12010]
table_m = [0.5, 0.55, 1.7, 3, 11, 32, 40, 41, 39]
table_T = [4000,5000,6000,7000,8000,9000,10000,11000,12000,13000,14000]
table_sigma = [0.031,0.27,2.05,6.06,12,19.9,29.6,41.1,54.1,67.7,81.5]

R = 0.35
le = 12
Lk = 187 * 10 ** (-6)
Ck = 268 * 10 ** (-6)
Rk = 0.25
Tw = 2000

T0 = 0
Rp = 0

res_I = []
res_U = []
res_t = []
res_Rp = []
res_IRp = []
res_T0 = []

def find_sigma(z, I, Tw):
	t0 = interpolation(I, 2, table_I, table_T0)
	global T0
	T0 = t0
	m = interpolation(I, 2, table_I, table_m)
	T = T0 + (Tw - T0) * z ** m
	sigma = interpolation(T, 2, table_T, table_sigma)
	return sigma

def find_Rp(I, Tw):
	a, b = 0, 1
	n = 100
	dz = (b - a) / n
	intgr = 0
	z = 0
	
	for j in range(n):
		intgr += z * dz * find_sigma(z, I, Tw)
		z += dz
	
	return le / (2 * pi * R * R * intgr)

def f(I, U):
	global Rp
	Rp = find_Rp(I, Tw)
	return (U - (Rk + Rp) * I) / Lk

def phi(I):
	return -I / Ck

def find_I_U(I, U, h):
	k1 = h * f(I, U)
	p1 = h * phi(I)
	
	k2 = h * f(I + k1 / 2, U + p1 / 2)
	p2 = h * phi(I + k1 / 2)
	
	k3 = h * f(I + k2 / 2, U + p2 / 2)
	p3 = h * phi(I + k2 / 2)
	
	k4 = h * f(I + k3, U + p3)
	p4 = h * phi(I + k3)
	
	return I + 1 / 6 * (k1 + 2 * k2 + 2 * k3 + k4), \
		U + 1 / 6 * (p1 + 2 * p2 + 2 * p3 + p4)

def find_position(x, data_x, number):
	left = 0
	right = number - 1
	
	if data_x[0] < data_x[-1]:
		if x < data_x[left] or x > data_x[right]: return -1
		while left + 1 != right:
			pos_x = int((left + right)/2)
			if data_x[pos_x] <= x:
				left = pos_x
			else:
				right = pos_x
		return left
	else:
		if x < data_x[right] or x > data_x[left]: return -1
		while left + 1 != right:
			pos_x = int((left + right)/2)
			if data_x[pos_x] >= x:
				left = pos_x
			else:
				right = pos_x
		return left

def find_range(pos_x, degree, data_x, data_y, number):
	half = int(degree / 2)
	left = pos_x - half
	right = pos_x + (degree - half)
	if left < 0:
		right += -left
		left = 0
	elif right > number - 1:
		left -= right - (number - 1)
		right = number - 1
	
	return data_x[left: right + 1], data_y[left: right + 1]

def find_polynomial(degree, nodes_x, nodes_y):
	div_diff, old_arr = [0] * (degree + 1), nodes_y
	new_arr = [0] * degree
	
	div_diff[0] = old_arr[0]
	for i in range(degree):
		for j in range(degree - i):
			new_arr[j] = (old_arr[j] - old_arr[j+1]) / \
			(nodes_x[j] - nodes_x[j+i+1])
		div_diff[i+1] = new_arr[0]
		old_arr = new_arr
	
	return lambda x: nwtn_polynom(x, degree, div_diff, nodes_x)

def nwtn_polynom(x, degree, div_diff, nodes_x):
	y = div_diff[0]
	x_pl = 1
	for i in range(1, degree + 1):
		x_pl *= (x - nodes_x[i-1])
		y += x_pl * div_diff[i]
	return y

def interpolation(x, degree, data_x, data_y):
	pos = find_position(x, data_x, len(data_x))
	nodes_x, nodes_y = find_range(pos,degree,data_x, data_y, len(data_x))
	f = find_polynomial(degree, nodes_x, nodes_y)
	return f(x)

def show_graph():
	/* Построение графиков */

def main():
	Uc = 1400
	I, t = 0, 0
	h = 10 ** (-7)
	
	while t < 1.5 * 10 ** (-3):
		I, Uc = find_I_U(I, Uc, h)
		t += h
		
		res_I.append(I)
		res_U.append(Uc)
		res_t.append(t)
		res_T0.append(T0)
		res_Rp.append(Rp)
	
	for i in range(len(res_Rp)):
		res_IRp.append(res_I[i] * res_Rp[i])
	
	show_graph()

if __name__ == '__main__':
	main()
\end{lstlisting}