\underline{\textbf{Вопросы при защите лабораторной работы}}\\

\begin{enumerate}
\item \textbf{Какие способы тестирования программы, кроме указанного в п.2, можете предложить ещё?}

Можно сравнить результаты работы программы с теми, которые получаются при аналитическом решении, так как представленная задача является слишком сложной для такого подхода, то можно упростить задачу, изменив некоторые условия. Например, при $R_k + R_p = 0$ система представляет собой идеальный колебательный контур, что позволяет оценить правильность работы по ряду характеристик.  

\item \textbf{Получите систему разностных уравнений для решения сформулированной задачи неявным методом трапеций. Опишите алгоритм реализации полученных уравнений.}
\begin{equation}
	\left\{
	\begin{array}{ccc}
	U_{n + 1} = U_n - \dfrac{h}{2 C_k}(I_n + I_{n + 1}) \\
	I_{n + 1} = I_n + \dfrac{h}{2 L_k}(U_n - (R_k + R_p(I_n)) I_n + U_{n + 1} - (R_k + R_p(I_{n + 1})) I_{n + 1})
\end{array}
\right.
\end{equation}

Сначала следует найти $R_p({I_{n + 1}})$, так как он имеет нелинейную зависимость от I, выраженный через интерполяцию таблицы значений, и поэтому выразить $I_{n + 1}$ через $R_p$ крайне сложно. В свою очередь, $R_p$ можно найти, подставив в него значение $I_{n + 1}$, найденное с помощью метода Эйлера. После этого, подставив все соответствующие значения, можно получить $I_{n + 1}$ по второй формуле из системы выше. И далее уже можно получить $U_{n + 1}$.

\item \textbf{Из каких соображений проводится выбор численного метода того или иного порядка точности, учитывая, что чем выше порядок точности метода, тем он более сложен и требует, как правило, больших ресурсов вычислительной системы?}

Это зависит от конкретной задачи, какую точность она требует. Увеличивать порядок точности нужно в том случае, если это существенно (относительно погрешности) повлияет на результат.


\item \textbf{Можно ли метод Рунге - Кутта применить для решения задачи, в которой часть условий задана на одной границе, а часть на другой? Например, напряжение по-прежнему задано при $t = 0$, т.е. $t = 0, U = U_0$ а ток задан в другой момент времени, к примеру, в конце импульса, т.е. при $t = T, I = I_T$. Какой можете предложить алгоритм вычислений?}

Сначала выберем ток в момент времени $t = 0$, и получим решение реализованным методом Рунге-Кутта для $I$ в $t = T$. Далее оценим невязку с заданной по условию величиной и повторим вычисления для другого тока в $t = 0$. Перебирая значения тока при $t = 0$, нужно свести невязку к минимальному значению. \\

Но решение поставленной задачи возможно только в случае, если заданные условия ведут только к одному решению. Например, если одним из условий будет то, что $I = 0$ в момент времени $t = 1$ мс, то будет существовать бесконечное количество решений, удовлетворяющих данному условию.
\end{enumerate}