\underline{\textbf{Вопросы при защите лабораторной работы}}\\

\begin{enumerate}
\item \textbf{Какие способы тестирования программы, кроме указанного в п.2, можете предложить ещё?}

\item \textbf{Получите систему разностных уравнений для решения сформулированной задачи неявным методом трапеций. Опишите алгоритм реализации полученных уравнений.}

\item \textbf{Из каких соображений проводится выбор численного метода того или иного порядка точности, учитывая, что чем выше порядок точности метода, тем он более сложен и требует, как правило, больших ресурсов вычислительной системы?}

\item \textbf{Можно ли метод Рунге - Кутта применить для решения задачи, в которой часть условий задана на одной границе, а часть на другой? Например, напряжение по-прежнему задано при $t = 0$, т.е. $t = 0, U = U_0$ а ток задан в другой момент времени, к примеру, в конце импульса, т.е. при $t = T, I = I_T$. Какой можете предложить алгоритм вычислений?}
\end{enumerate}