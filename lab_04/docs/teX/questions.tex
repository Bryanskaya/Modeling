\underline{\textbf{Вопросы при защите лабораторной работы}}\\
\begin{enumerate}
\item \textit{Приведите результаты тестирования программы (графики, общие соображения, качественный анализ)}

\begin{enumerate}
	\item $F_0(t) = 0$  
	
	Из-за того, что тепловой поток равен нулю, температура равна $T_0$ для всех $x$.
	\begin{figure}[h]
	\begin{center}
		{\includegraphics[height=6.5cm, width = 10cm]{../pictures/Figure_3}}
		\caption{Тест 1}
	\end{center}
	\end{figure}

	\item Изменение теплопроводности (увеличение)
	
	Увеличение теплопроводности должно вызвать более равномерное распределение тепла в температурном поле. По результатам видно, что значение температуры стало меньше, и графики стали более пологими.
	\begin{figure}[h]
		\begin{center}
			{\includegraphics[height=6.5cm, width = 10cm]{../pictures/Figure_5}}
			\caption{Тест 2}
		\end{center}
	\end{figure}

\newpage

	\item Изменение теплоёмкости (увеличение)

	Увеличение теплоёмкости должно вызвать уменьшение температуры, и как следствие, во всем температурном поле также должна ухудшиться теплопроводность. Поэтому распределение температуры в поле должно стать менее равномерным. Также должно увеличиться время нагрева. 
	\begin{figure}[h]
		\begin{center}
			{\includegraphics[height=6.5cm, width = 10cm]{../pictures/Figure_6}}
			\caption{Тест 3}
		\end{center}
	\end{figure}

\newpage

	\item $F_0(t) < 0$
	
	В такой ситуации происходит съём тепла.
	\begin{figure}[h]
		\begin{center}
			{\includegraphics[height=6.5cm, width = 10cm]{../pictures/Figure_4}}
			\caption{Тест 4}
		\end{center}
	\end{figure}

	
\end{enumerate}
\end{enumerate}








