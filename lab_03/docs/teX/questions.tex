\underline{\textbf{Вопросы при защите лабораторной работы}}\\
\begin{enumerate}
\item Какие способы тестирования программы можно предложить?

\item Получите простейший разностный аналог нелинейного краевого условия при $x = l$
$x = l, -k(l)\frac{dT}{dx} = \alpha_N(T(l) - T_0) + \phi(T).$
где $\phi(T)$ - заданная функция.
Производную аппроксимируйте односторонней разностью

\item Опишите алгоритм применения метода прогонки, если при $x = 0$ краевое условие квазилинейное (как в настоящей работе), а при $x = l$, как в п.2

\item  Опишите алгоритм определения единственного значения сеточной функции $y_p$ в одной заданной точке p. Использовать встречную прогонку, т.е. комбинацию правой и левой прогонок (лекция №8). Оба краевых условия линейные.

\end{enumerate}