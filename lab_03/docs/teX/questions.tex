\underline{\textbf{Вопросы при защите лабораторной работы}}\\
\begin{enumerate}
\item \textit{Какие способы тестирования программы можно предложить?}

\qquadУстановив $F_0 = 0$, можно протестировать уровень температуры, который должен с небольшой погрешностью равняться $T_0$, также при данном условии можно изменить сам $T_0$, чтобы удостовериться, что это происходит при всех температурах.

\qquadПроверка поведения графика при отрицательном/положительном $F_0$ (значение температуры должны строго возрастать/убывать).

\qquadИзменение $\alpha$. Увеличение должно приводить к тому, что $T(l) \rightarrow T_0$.

\qquadПроверка сходимости решения при различных начальных распределениях и $\varepsilon$ (коэффициент релаксации).

\item \textit{Получите простейший разностный аналог нелинейного краевого условия при
$x = l, -k(l)\frac{dT}{dx} = \alpha_N(T(l) - T_0) + \phi(T).$
где $\phi(T)$ - заданная функция.
Производную аппроксимируйте односторонней разностью.}

Применяя аппроксимацию односторонней разностью получаем:

\begin{equation}
	-k(l)\dfrac{t_N - t_{N-1}}{h} = \alpha_N(t_N - T_0) + \phi(t_N)
\end{equation}

\item \textit{Опишите алгоритм применения метода прогонки, если при $x = 0$ краевое условие квазилинейное (как в настоящей работе), а при $x = l$, как в п.2}

\qquadПрименяется правая прогонка, то есть, коэффициенты определяются слева направо, значение функции справа налево. Так как правое краевое условие зависит от $T$, прогонка используется в несколько итераций, в каждой итерации $s$ в качестве значения $T$ используется $t_{\varepsilon n}^{s-1}$. При вычислении значения $t_N$ используется полученные прогонкой коэффициенты и $t_{\varepsilon N}^{s-1}$. 

\item  \textit{Опишите алгоритм определения единственного значения сеточной функции $y_p$ в одной заданной точке p. Использовать встречную прогонку, т.е. комбинацию правой и левой прогонок (лекция №8). Оба краевых условия линейные.}

\qquadЛевая прогонка:
\begin{equation}
	y_n = \varepsilon_{n+1} y_{n+1} + \eta_{n+1}
\end{equation}

\qquadПравая прогонка:
\begin{equation}
	y_n = \alpha_{n-1} y_{n-1} + \beta_{n-1}
\end{equation}

\qquadВ правой прогонке при $n = p$ получаем выражение: 
\begin{equation}
	-\alpha_{p - 1}y_{p - 1} + y_p = \beta_{p - 1}
\end{equation}
Данные выражения можно применить как правое краевое условие для левой прогонки: 
	\begin{equation*}
	\begin{cases}
		K_p = - \alpha_{p - 1} \\ 
		M_p = 1\\
		P_p = \beta_{p - 1}\\
	\end{cases}
\end{equation*}

Применяя составленное краевое условие с коэффициентами, полученными левой прогонкой получаем:
\begin{equation}
	y_p = \frac{P_p - K_p \cdot \eta_p}{M_p + K_p \cdot \varepsilon_p} = \frac{\beta_{p - 1} + \alpha_{p - 1} \cdot \eta_p}{1 - \alpha_{p - 1} \cdot \varepsilon_p}
\end{equation}

\qquadПо условию оба краевых условия - линейные, поэтому прогонку можно начинать с обеих сторон. В левой прогонке коэффициенты нужно вычислить от 0 до $p$, в правой от $N$ до $p - 1$.

\end{enumerate}








