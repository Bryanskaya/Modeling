\chapter*{Задание}

\textbf{Тема. } Программно- алгоритмическая реализация моделей на основе ОДУ второго порядка
с краевыми условиями II и III рода.\\

\textbf{Цель работы. } Получение навыков разработки алгоритмов решения краевой задачи при
реализации моделей, построенных на ОДУ второго порядка. \\

\textbf{Исходные данные. }\\
\begin{enumerate}
	\item Задана математическая модель. \\
	Квазилинейное уравнение для функции T(x)
	
	\begin{equation}\label{formula1}
		\frac{d}{dx}\bigg(\lambda(T)\frac{dT}{dx}\bigg) - 4 \cdot k(T) \cdot n_p^{2} \cdot \sigma \cdot (T^4 - T_0^4) = 0
	\end{equation}

	Краевые условия:
	\begin{equation}\label{formula2}
		\left\{
		\begin{array}{ccc}
			x = 0, -\lambda(T(0))\frac{dT}{dx} = F_0,\\
			x = l, -\lambda(T(l))\frac{dT}{dx} = \lambda(T(l) - T_0) \\
		\end{array}
		\right.
	\end{equation}

	\item Функции $\lambda(T), k(T)$ заданы таблицей
	\begin{table}[ph!]\label{table_1}
		\caption{}
		\centering
		\begin{tabular}{|c|c|c|c|c|}
			\hline
			$T, K$ & $\lambda$, Вт/(см К) & & $T, K$ & $k,$ см(-1)\\
			\hline
			300 & $1.36 \cdot 10^{-2}$ && 293 & $2.0 \cdot 10^{-2}$\\
			\hline
			500 & $1.63 \cdot 10^{-2}$ && 1278 & $5.0 \cdot 10^{-2}$\\
			\hline
			800 & $1.81 \cdot 10^{-2}$ && 1528 & $7.8 \cdot 10^{-2}$ \\
			\hline
			1100 & $1.98 \cdot 10^{-2}$ && 1677 & $1.0 \cdot 10^{-1}$ \\
			\hline
			2000 & $2.50 \cdot 10^{-2}$ && 2000 & $1.3 \cdot 10^{-1}$ \\
			\hline
			2400 & $2.74 \cdot 10^{-2}$ && 2400 & $2.0 \cdot 10^{-1}$ \\
			\hline
	
		\end{tabular}
	\end{table}

	\item Разностная схема с разностным краевым условием при $x = 0$. Получена в Лекции №7, и может быть использована в данной работе. Самостоятельно надо получить интегро-интерполяционным методом разностный аналог краевого условия при $x = l$, точно
	так же, как это было сделано применительно к краевому условию при $x = 0$ в указанной 
	лекции. Для этого надо проинтегрировать на отрезке $[x_{N-\frac{1}{2}}, x_N]$ записанное выше уравнение \ref{formula1} и учесть, что поток $F_N = \alpha_N(y_N - T_0)$, а $F_{N - \frac{1}{2}} = \chi_{N - \frac{1}{2}}(\frac{y_{N-1} - y_N}{h})$
	
	\item Значения параметров для отладки (все размерности согласованы)
	$n_p$ = 1.4 – коэффициент преломления,
	l = 0.2 см – толщина слоя,
	$T_0$ = 300К – температура окружающей среды,
	$\sigma$ = $5.668 \cdot 10_{-12}$ Вт/(см2K4) - постоянная Стефана- Больцмана,
	$F_0$ = 100 Вт/см2 - поток тепла,
	$\alpha$ = 0.05 Вт/(см2 К) – коэффициент теплоотдачи. 
	
	\item  Выход из итераций организовать по температуре и по балансу энергии, т.е.
	\begin{equation}\label{formula3}
		max \bigg|\frac{y_n^s - y_n^{s-1}}{y_n^s} \bigg| \leq \varepsilon_1, n = 0, 1, ..., N
	\end{equation}

	\begin{equation}\label{formula4}
		max \bigg|\frac{f_1^s - y_2^s}{f_1^s} \bigg| \leq \varepsilon_2
	\end{equation}
	
	где 
	\begin{equation}\label{formula5}
		f_1 = F_0 - \alpha(T(l) - T_0)
	\end{equation}
	и
	\begin{equation}\label{formula6}
		f_2 = 4n_p^2 \sigma \int_0^l k(T(x))(T^4(x) - T_0^4)dx
	\end{equation}
	
\end{enumerate}

\textbf{Результаты работы.}
\begin{enumerate}
	\item Представить разностный аналог краевого условия при $x = l$
	и его краткий вывод интегро -интерполяционным методом.
	\item График зависимости температуры T(x) от координаты x при заданных выше параметрах.\\
	Выяснить, как сильно зависят результаты расчета T(x) и необходимое для этого количество итераций от начального распределения температуры и шага сетки.
	\item График зависимости T(x) при $F_0 = -10$ Вт/см2. 
	
	\qquad \textit{Справка.} При отрицательном тепловом потоке слева идет съем тепла, поэтому
	производная T(x) должна быть положительной.
	\item График зависимости T(x) при увеличенных значениях $\alpha$ (например, в 3 раза). Сравнить с п.2.
	
	\qquad \textit{Справка.} При увеличении теплосъема и неизменном потоке $F_0$ уровень температур T(x)
	должен снижаться, а градиент увеличиваться.
	\item График зависимости T(x) при $F_0 = 0$.
	
	\qquad \textit{Справка.} В данных условиях тепловое нагружение отсутствует, причин для
	нагрева нет, температура стержня должна быть равна температуре окружающей
	среды $T_0$ (разумеется с некоторой погрешностью, определяемой приближенным
	характером вычислений).
	\item Для указанного в задании исходного набора параметров привести данные по балансу
	энергии, т.е. значения величин
	\begin{equation}\label{formula7}
		f_1 = F_0 - \alpha(T(l) - T_0)
	\end{equation}
	и
	\begin{equation}\label{formula8}
		f_2 = 4n_p^2 \sigma \int_0^l k(T(x))(T^4(x) - T_0^4)dx
	\end{equation}
	Каковы использованные в работе значения точности выхода из итераций
	$\varepsilon_1$ (по температуре) и
	$\varepsilon_2$ (по балансу энергии)?

\end{enumerate}



