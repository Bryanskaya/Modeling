\underline{\textbf{Вопросы при защите лабораторной работы}}\\

\begin{enumerate}
\item \textbf{Укажите интервалы значений аргумента, в которых можно считать решением заданного уравнения каждое из первых 4-х приближений Пикара. Точность результата оценивать до второй цифры после запятой. Объяснить свой ответ.}

\item \textbf{Пояснить, каким образом можно доказать правильность полученного результата при фиксированном значении аргумента в численных методах.}

В силу того, что численные методы зависят от величины шага, то изменяя его, можно прийти к наиболее точному результату при фиксированном значении аргумента. Как только результат перестаёт отличатся от результатов, полученных ранее, то можно сделать вывод о том, что корректный результат получен.

\item \textbf{Каково значение функции при x=2, т.е. привести значение u(2).}

При $x = 2$ и шаге $10^{-6}$ методом Рунге-Кутта было получено значение функции равное $317.82$.
\end{enumerate}