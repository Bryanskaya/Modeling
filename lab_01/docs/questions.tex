\underline{\textbf{Вопросы при защите лабораторной работы}}\\

\begin{enumerate}
\item \textbf{Укажите интервалы значений аргумента, в которых можно считать решением заданного уравнения каждое из первых 4-х приближений Пикара. Точность результата оценивать до второй цифры после запятой. Объяснить свой ответ.}

Левая граница у каждого из четырёх интервалов равна 0. Для того, чтобы определить правую, нужно сравнить полученные значения для разных приближений или с численными методами. Сравнивая значение текущего приближения со значением, которое было расчитано методом более высокого порядка, можно определить правую границу (это последнее значение, при котором наблюдается совпадение результатов).\\

Проведя анализ были получены следующие интервалы:
\begin{itemize}
	\item Первое приближение: 	 [0; 0.88]
	\item Второе приближение:    [0; 1.17]
	\item Третье приближение: 	 [0; 1.40]
	\item Четвёртое приближение: [0; 1.47]
\end{itemize}

\item \textbf{Пояснить, каким образом можно доказать правильность полученного результата при фиксированном значении аргумента в численных методах.}

В силу того, что численные методы зависят от величины шага, то изменяя его, можно прийти к наиболее точному результату при фиксированном значении аргумента. Как только результат перестаёт отличатся от результатов, полученных ранее, то можно сделать вывод о том, что корректный результат получен.

\item \textbf{Каково значение функции при x=2, т.е. привести значение u(2).}

При $x = 2$ и шаге $10^{-6}$ было получено значение функции равное $317.82$.
\end{enumerate}