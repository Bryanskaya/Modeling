\documentclass[12pt]{report}
\usepackage[utf8]{inputenc}
\usepackage[russian]{babel}
\usepackage{setspace} % для междустрочного интервала
\onehalfspacing % 1.5 интервал между строками

\usepackage[left=30mm, top=20mm, right=20mm, bottom=20mm, nohead, footskip=7mm]{geometry}

\usepackage{titlesec, blindtext, color} 
\definecolor{gray75}{gray}{0.75}
\newcommand{\hsp}{\hspace{20pt}}
\titleformat{\chapter}[hang]{\Large\bfseries}{\thechapter{. }}{0pt}{\Large\bfseries}
\titlespacing{\chapter}{-5pt}{-30pt}{12pt} % отступ заголовка сверху
\titleformat{\section}[hang]{\large\bfseries}{\thesection{. }}{0pt}{\large\bfseries}

\makeatletter % список литературы
\def\@biblabel#1{#1. }
\makeatother

% Ссылки
\usepackage{hyperref}

% Возможность вставки pdf страниц
\usepackage{pdfpages}

% Листинги
\usepackage{listings}

% Для возможности переноса строк в equation, только надо еще и окружение \begin{gathered} сделать
\usepackage{amsmath}

\lstset{
	language = c++,
	extendedchars=\true,
	basicstyle=\small\sffamily,
	numbers=left,
	numberstyle=\tiny,
	stepnumber=1,
	numbersep=5pt,
	showspaces=false,            % показывать или нет пробелы специальными отступами
	showstringspaces=false,
	showtabs=false,
	frame=single,
	tabsize=2,
	captionpos=t,
	breaklines=true,
	breakatwhitespace=false,
	escapeinside={\#*}{*)},
	keepspaces=true
}

% Чтобы вместо : в подписях было -
\RequirePackage{caption}
\DeclareCaptionLabelSeparator{defffis}{ — }
\captionsetup{justification=centering,labelsep=defffis}

\usepackage{pgfplots}
\usepackage{pgfplotstable}
\pgfplotsset{compat=1.9}


\usepackage{csvsimple} %
\usepackage{datatool}




\begin{document}
	\renewcommand\bibname{Список литературы}
	
	\includepdf[pages=1]{title.pdf}
	
	\newpage
	
	\section*{Задание}

\qquad\textbf{Тема. } Программная реализация приближенного аналитического метода и численных алгоритмов первого и второго порядков точности при решении задачи Коши для ОДУ.\\

\textbf{Цель работы. } Получение навыков решения задачи Коши для ОДУ методами Пикара и явными методами первого порядка точности (Эйлера) и второго порядка точности (Рунге-Кутта).
\\

\textbf{Исходные данные. }\\
ОДУ, не имеющее аналитического решения:\\
\begin{equation}\label{formula1}
	\left\{
	\begin{array}{ccc}
		u'(x) = x^2 + u^2,\\
		u(0) = 0\\
	\end{array}
	\right.
\end{equation}\\

\textbf{Результат работы программы.} Таблица, содержащая значения аргумента с заданным шагом в интервале [0, $x_{max}$] и результаты расчета функции $u(x)$ в приближениях Пикара (от 1-го до 4-го), а также численными методами. Границу интервала $x_{max}$ выбирать максимально возможной из условия, чтобы численные методы обеспечивали точность вычисления решения уравнения $u(x)$ до второго знака после запятой. \\
	\section*{Описание алгоритмов}
\textbf{Задача Коши}\\

Общее решение дифференциального уравнения $n$-ого порядка зависит от $n$ констант. Требуется задать $n$ дополнительных условий:
\begin{equation}\label{formula2}
	u(x) = \phi(x, c_1, c_2, ... c_n)
\end{equation}

В задаче Коши все дополнительные условия задаются в одной точке $\xi$:
\begin{equation}\label{formula3}
	u_k(\xi) = \eta_k, k = 1,...n
\end{equation}

Задачу Коши можно решить с помощью следующих алгоритмов. \\

	\subsection*{Приближённый аналитический метод Пикара}
\begin{equation}\label{formula4}
	\left\{
	\begin{array}{ccc}
		u'(x) = f(x, u),\\
		u(\xi) = \eta\\
	\end{array}
	\right.
\end{equation}

\begin{equation}\label{formula5}
	u(x) = \eta + \int\limits_\xi^x f(t, u(t))dt
\end{equation}

Получается, что
\begin{equation}\label{formula6}
	y^{(s)}(x) = \eta + \int\limits_\xi^x f(t, y^{(s - 1)}(t))dt
\end{equation}

\begin{equation}\label{formula7}
	y^{(0)} = \eta
\end{equation}

Найдём 1, 2, 3 и 4 приближение для (\ref{formula1}).\\
\begin{multline}\label{formula8}
	\shoveright
	{
		y^{(1)} = 0 + \int\limits_0^x t^2dt = \frac{t^3}{3}\bigg|_0^x = \frac{x^3}{3}
	}
\end{multline}


\begin{multline}\label{formula9}
	\shoveright
	{
		y^{(2)} = 0 + \int\limits_0^x [(\frac{t^3}{3})^2 + t^2]dt = \frac{t^7}{63}\bigg|_0^x + \frac{t^3}{3}\bigg|_0^x = \frac{x^7}{63} + \frac{x^3}{3}
	}
\end{multline}

\begin{multline}\label{formula10}
	\shoveright
	{
	\begin{gathered}
		y^{(3)} = 0 + \int\limits_0^x [(\frac{t^3}{3} + \frac{t^7}{63})^2 + t^2]dt = \frac{t^{15}}{15\cdot63^2}\bigg|_0^x + \frac{2\cdot t^{11}}{3\cdot63\cdot11}\bigg|_0^x + \frac{t^7}{63}\bigg|_0^x + \frac{t^3}{3}\bigg|_0^x = \\
		= \frac{x^{15}}{59535} + \frac{2\cdot x^{11}}{2079} + \frac{x^7}{63} + \frac{x^3}{3}
	\end{gathered}
	}
\end{multline}

\begin{multline}\label{formula11}
	\shoveright
	{
	\begin{gathered}
		y^{(4)} = 0 + \int\limits_0^x [(\frac{t^{15}}{59535} + \frac{2\cdot t^{11}}{2079} + \frac{t^7}{63} + \frac{t^3}{3})^2 + t^2]dt
		= \frac{x^{31}}{109 876 902 975} + \frac{4\cdot x^{27}}{3 341 878 155} + \\ + \frac{4\cdot x^{23}}{99 411 543} + \frac{2\cdot x^{23}}{86 266 215} + \frac{2\cdot x^{19}}{3 393 495} + \frac{4\cdot x^{19}}{2 488 563} + \frac{4\cdot x^{15}}{93 555} + \frac{x^{15}}{59 535} + \frac{2\cdot x^{11}}{2079} + \frac{x^7}{63} + \frac{x^3}{3}
	\end{gathered}
	}
\end{multline}

Реализация представлена на листинге \ref{code1}.\\


%\pgfplotstabletypeset[
%col sep=comma,
%string type,
%columns/x/.style={column type={|c|}},
%columns/7/.style={column type={c|}},
%every head row/.style={before row=\hline, after row=\hline},
%every last row/.style={after row=\hline},
%]{result.csv}

%\DTLsetseparator{,}      % разделитель ячеек
%\DTLloaddb{satellites}{result.csv}

%\begin{table}
%	\DTLdisplaydb{satellites}
%\end{table}






	\underline{\textbf{Вопросы при защите лабораторной работы}}\\

\begin{enumerate}
\item \textbf{Какие способы тестирования программы, кроме указанного в п.2, можете предложить ещё?}

\item \textbf{Получите систему разностных уравнений для решения сформулированной задачи неявным методом трапеций. Опишите алгоритм реализации полученных уравнений.}

\item \textbf{Из каких соображений проводится выбор численного метода того или иного порядка точности, учитывая, что чем выше порядок точности метода, тем он более сложен и требует, как правило, больших ресурсов вычислительной системы?}

\item \textbf{Можно ли метод Рунге - Кутта применить для решения задачи, в которой часть условий задана на одной границе, а часть на другой? Например, напряжение по-прежнему задано при $t = 0$, т.е. $t = 0, U = U_0$ а ток задан в другой момент времени, к примеру, в конце импульса, т.е. при $t = T, I = I_T$. Какой можете предложить алгоритм вычислений?}
\end{enumerate}
	\newpage
	
%	\chapter{Аналитическая часть}
	%\input{analytical_part.tex}
%	\newpage
	
	
%	\chapter*{Заключение}
%	\input{conclusion.tex}
%	\newpage
	
	\begin{thebibliography}{2}
		\addcontentsline{toc}{chapter}{Список литературы}
		\bibitem{Paral_methods}Иванов, К. К. Принципы разработки параллельных методов / К. К. Иванов, С. А. Раздобудько, Р. И. Ковалев. — Текст : непосредственный // Молодой ученый. — 2017. — № 3 (137). — С. 30-32. — URL: https://moluch.ru/archive/137/38412/ (дата обращения: 21.10.2020).
		\bibitem{Kormen} Кормен, Томас Х. и др Алгоритмы: построение и анализ, 3-е изд. : Пер. с англ. - М. : ООО "И.Д. Вильямс", 2018. - 1328 с. : ил. - Парал. тит. англ. -  ISBN 978-5-8459-2016-4 (рус.).
		\bibitem{thread} Документация по Стандартной библиотекн языка С++ thread [Электронный ресурс]. Режим доступа: https://docs.microsoft.com/ru-ru/cpp/standard-library/thread?view=vs-2019, свободный (дата обращения 22.10.2020)
		\bibitem{mutex} Документация по Стандартной библиотекн языка С++ mutex [Электронный ресурс]. Режим доступа: https://docs.microsoft.com/ru-ru/cpp/standard-library/mutex?view=vs-2019, свободный (дата обращения 22.10.2020)
		\bibitem{Visual} Документация по Visual Studio 2019 [Электронный ресурс]. Режим доступа: https://docs.microsoft.com/ru-ru/visualstudio/windows/?view=vs-2019, свободный (дата обращения: 21.10.2020)
		\bibitem{Query} QueryPerformanceCounter function [Электронный ресурс]. Режим доступа: https://docs.microsoft.com/en-us/windows/win32/api/profileapi/nf-profileapi-queryperformancecounter, свободный (дата обращения: 22.10.2020).
	\end{thebibliography}
\end{document}
